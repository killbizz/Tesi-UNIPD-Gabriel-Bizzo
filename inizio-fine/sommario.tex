% !TEX encoding = UTF-8
% !TEX TS-program = pdflatex
% !TEX root = ../tesi.tex

%**************************************************************
% Sommario
%**************************************************************
\cleardoublepage
\phantomsection
\pdfbookmark{Sommario}{Sommario}
\begingroup
\let\clearpage\relax
\let\cleardoublepage\relax
\let\cleardoublepage\relax

\chapter*{Sommario}

Il presente documento descrive il lavoro svolto durante il periodo di stage, della durata di circa trecentoventi ore, dal laureando Gabriel Bizzo presso l'azienda Sync Lab S.r.l. \\
Il lavoro di stage si inserisce in un progetto, denominato Voting-Online, che consiste nello sviluppo di un'applicazione web nell'ambito e-Voting, ossia un sistema per permettere di eseguire delle votazioni online e il relativo monitoraggio da parte di un amministratore. \\
Gli obbiettivi da raggiungere erano molteplici.\\
In primo luogo era richiesto lo studio del linguaggio di programmazione Java e del \gls{frameworkg} Spring, da effettuare come semplice approfondimento ma in seguito utilizzato per lo sviluppo del \gls{backend} del progetto Voting-Online. \\
In secondo luogo era richiesto lo studio delle tecnologie necessarie allo sviluppo del \gls{frontend} del succitato progetto. In particolare i linguaggi di programmazione Javascript e Typescript, il \gls{frameworkg} Angular e la libreria React. \\
Infine era richiesta la progettazione, e successiva implementazione, delle maschere  necessarie al funzionamento dell'applicazione web, con lo sviluppo di un relativo documento tecnico.

%\vfill
%
%\selectlanguage{english}
%\pdfbookmark{Abstract}{Abstract}
%\chapter*{Abstract}
%
%\selectlanguage{italian}

\endgroup			

\vfill

