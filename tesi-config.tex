%**************************************************************
% file contenente le impostazioni della tesi
%**************************************************************

% 1 - Introduzione - Obiettivi

\newcommand{\obiettiviObbligatori}{
	 \item \underline{\textit{O01}}: Acquisizione competenze sulle tematiche sopra descritte;
	 \item \underline{\textit{O02}}: Capacità di raggiungere gli obiettivi richiesti in autonomia seguendo il cronoprogramma;
	 \item \underline{\textit{O03}}: Portare a termine le implementazioni previste con una percentuale di superamento pari all'80\%.
	 
}

\newcommand{\obiettiviDesiderabili}{
	 \item \underline{\textit{D01}}: Portare a termine le implementazioni previste con una percentuale di superamento pari al 100\%.
}

\newcommand{\obiettiviFacoltativi}{
	 \item \underline{\textit{F01}}: Riuscire a renderizzare le maschere lato server usando Next.Js/React.
}

% 2 - Descrizione dello stage - Pianificazione del lavoro

\newcommand{\prospettoSettimanale}{
     % Personalizzare indicando in lista, i vari task settimana per settimana
     % sostituire a XX il totale ore della settimana
    \begin{itemize}
        \item \textbf{Prima Settimana (40 ore)}
        \begin{itemize}
            \item Incontro con persone coinvolte nel progetto per discutere i requisiti e le richieste relativamente al sistema da sviluppare; 
            \item Verifica credenziali e strumenti di lavoro assegnati;
            \item Analisi del progetto da svolgere;
            \item Ripasso del linguaggio Java SE;
            \item Ripasso concetti Web (Servlet, servizi Rest, Json, ecc.).
        \end{itemize}
        \item \textbf{Seconda Settimana (40 ore)} 
        \begin{itemize}
            \item Studio principi generali di Spring Core (IOC, Dependency Injection);
            \item Studio di SpringBoot.
        \end{itemize}
        \item \textbf{Terza Settimana (40 ore)} 
        \begin{itemize}
            \item Studio Spring DataRest;
            \item Studio Spring Data/JPA.
        \end{itemize}
        \item \textbf{Quarta Settimana (40 ore)} 
        \begin{itemize}
            \item Ripasso linguaggio Javascript e studio TypeScript;
            \item Studio del \gls{frameworkg} Angular.
        \end{itemize}
        \item \textbf{Quinta Settimana (40 ore)} 
        \begin{itemize}
            \item Analisi e studio del progetto Voting-on-line;
            \item Progettazione ed implementazione della nuova maschera di login in Angular;
            \item Progettazione ed implementazione nuova maschera "Voto" in Angular.
        \end{itemize}
        \item \textbf{Sesta Settimana (40 ore)} 
        \begin{itemize}
            \item Ripasso \gls{frameworkg} React;
            \item Progettazione ed implementazione della nuova maschera di login in React;
            \item Progettazione ed implementazione nuova maschera "Voto" in React.
        \end{itemize}
        \item \textbf{Settima Settimana (40 ore)} 
        \begin{itemize}
            \item Analisi comparativa dei due \gls{frameworkg} utilizzati.
        \end{itemize}
        \item \textbf{Ottava Settimana (40 ore)} 
        \begin{itemize}
            \item Considerazioni finali e stesura elaborato.
        \end{itemize}
    \end{itemize}
}

% 5 - Codeword 

\newcommand{\codeword}[1]{%
\texttt{\textcolor{magenta}{#1}}%
}

%**************************************************************
% Frontespizio
%**************************************************************

% Autore
\newcommand{\myName}{Gabriel Bizzo}                                    
\newcommand{\myTitle}{Confronto tra Angular e React nella realizzazione di una piattaforma di e-Voting}

% Tipo di tesi                   
\newcommand{\myDegree}{Tesi di laurea triennale}

% Università             
\newcommand{\myUni}{Università degli Studi di Padova}

% Facoltà       
\newcommand{\myFaculty}{Corso di Laurea in Informatica}

% Dipartimento
\newcommand{\myDepartment}{Dipartimento di Matematica "Tullio Levi-Civita"}

% Titolo del relatore
\newcommand{\profTitle}{Prof.}

% Relatore
\newcommand{\myProf}{Gilberto Filè}

% Luogo
\newcommand{\myLocation}{Padova}

% Anno accademico
\newcommand{\myAA}{2020-2021}

% Data discussione
\newcommand{\myTime}{Settembre 2021}


%**************************************************************
% Impostazioni di impaginazione
% see: http://wwwcdf.pd.infn.it/AppuntiLinux/a2547.htm
%**************************************************************

\setlength{\parindent}{14pt}   % larghezza rientro della prima riga
\setlength{\parskip}{0pt}   % distanza tra i paragrafi


%**************************************************************
% Impostazioni di biblatex
%**************************************************************
\bibliography{bibliografia} % database di biblatex 

\defbibheading{bibliography} {
    \cleardoublepage
    \phantomsection 
    \addcontentsline{toc}{chapter}{\bibname}
    \chapter*{\bibname\markboth{\bibname}{\bibname}}
}

\setlength\bibitemsep{1.5\itemsep} % spazio tra entry

\DeclareBibliographyCategory{opere}
\DeclareBibliographyCategory{web}

\addtocategory{opere}{womak:lean-thinking}
\addtocategory{web}{site:agile-manifesto}

\defbibheading{opere}{\section*{Riferimenti bibliografici}}
\defbibheading{web}{\section*{Siti Web consultati}}


%**************************************************************
% Impostazioni di caption
%**************************************************************
\captionsetup{
    tableposition=top,
    figureposition=bottom,
    font=small,
    format=hang,
    labelfont=bf
}

%**************************************************************
% Impostazioni di glossaries
%**************************************************************

%**************************************************************
% Acronimi
%**************************************************************
\renewcommand{\acronymname}{Acronimi e abbreviazioni}

\newacronym[description={\glslink{apig}{Application Program Interface}}]
    {api}{API}{Application Program Interface}
    
\newacronym[description={\glslink{guig}{Graphical User Interface}}]
    {gui}{GUI}{Graphical User Interface}

\newacronym[description={\glslink{umlg}{Unified Modeling Language}}]
    {uml}{UML}{Unified Modeling Language}

\newacronym[description={\glslink{backendg}{Backend}}]
    {backend}{BE}{Backend}
    
\newacronym[description={\glslink{frontendg}{Frontend}}]
    {frontend}{FE}{Frontend}
    
\newacronym[description={\glslink{inversion-of-controlg}{Inversion of Control}}]
    {inversion of control}{IoC}{Inversion of Control}
    
\newacronym[description={\glslink{restg}{Representational State Transfer}}]
    {rest}{REST}{Representational State Transfer}
    
\newacronym[description={\glslink{dbmsg}{Database Management System}}]
    {dbms}{DBMS}{Database Management System}

%**************************************************************
% Glossario
%**************************************************************
%\renewcommand{\glossaryname}{Glossario}

\newglossaryentry{apig}
{
    name=\glslink{api}{API},
    text=Application Program Interface,
    sort=api,
    description={in informatica con il termine \emph{Application Programming Interface API} (ing. interfaccia di programmazione di un'applicazione) si indica ogni insieme di procedure disponibili al programmatore, di solito raggruppate a formare un set di strumenti specifici per l'espletamento di un determinato compito all'interno di un certo programma. La finalità è ottenere un'astrazione, di solito tra l'hardware e il programmatore o tra software a basso e quello ad alto livello semplificando così il lavoro di programmazione}
}

\newglossaryentry{guig}
{
    name=\glslink{gui}{GUI},
    text=Graphical User Interface,
    sort=Graphical User Interface,
    description={L'interfaccia grafica, nota anche come GUI (dall'inglese Graphical User Interface), in informatica è un tipo di interfaccia utente che consente l'interazione uomo-macchina in modo visuale utilizzando rappresentazioni grafiche (es. widget) piuttosto che utilizzando i comandi tipici di un'interfaccia a riga di comando}
}

\newglossaryentry{umlg}
{
    name=\glslink{uml}{UML},
    text=UML,
    sort=uml,
    description={in ingegneria del software \emph{UML, Unified Modeling Language} (ing. linguaggio di modellazione unificato) è un linguaggio di modellazione e specifica basato sul paradigma object-oriented. L'\emph{UML} svolge un'importantissima funzione di ``lingua franca'' nella comunità della progettazione e programmazione a oggetti. Gran parte della letteratura di settore usa tale linguaggio per descrivere soluzioni analitiche e progettuali in modo sintetico e comprensibile a un vasto pubblico}
}

\newglossaryentry{backendg}
{
    name=\glslink{backend}{BE},
    text=Backend,
    sort=backend,
    description={in informatica con il termine \emph{Backend} si intende l'insieme delle applicazioni e dei programmi con cui l'utente non interagisce direttamente ma che sono essenziali al funzionamento del sistema}
}

\newglossaryentry{frontendg}
{
    name=\glslink{frontend}{FE},
    text=Frontend,
    sort=frontend,
    description={in informatica con il termine \emph{Frontend} si intende la parte visibile all'utente di un programma e con cui egli può interagire, tipicamente un'interfaccia utente}
}



\newglossaryentry{Entita'-Relazione}
{
    name=\glslink{Entita'-Relazione}{ER},
    text=Entita'-Relazione,
    sort=Entita'-Relazione,
    description={In informatica, nell'ambito della progettazione dei database, il modello entità-relazione (o modello entità-associazione; più comune modello E-R) è un modello teorico per la rappresentazione concettuale e grafica dei dati a un alto livello di astrazione}
}

\newglossaryentry{inversion-of-controlg}
{
    name=\glslink{inversion of control}{IoC},
    text=Inversion of Control,
    sort=Inversion of Control,
    description={L'\emph{Inversion of Control} è un principio architetturale basato sul concetto di invertire il controllo del flusso di sistema rispetto alla programmazione tradizionale. Nella programmazione tradizionale la logica di tale flusso è definita esplicitamente dallo sviluppatore, che si occupa tra le altre cose di tutte le operazioni di creazione, inizializzazione ed invocazione dei metodi degli oggetti. L'IoC invece inverte il controllo}
}

\newglossaryentry{endpointg}
{
    name=\glslink{endpoint}{Endpoint},
    text=Endpoint,
    sort=Endpoint,
    description={Una comunicazione \emph{endpoint} è un nodo di comunicazione rilevabile la cui portata può essere variata per restringere o ampliare la zona di ricerca. Gli endpoint favoriscono uno strato di astrazione programmabile per cui sistemi software e/o sottosistemi possono comunicare tra di loro, inoltre i mezzi di comunicazione sono disaccoppiati dai sottosistemi di comunicazione}
}

\newglossaryentry{restg}
{
    name=\glslink{rest}{REST},
    text=Representational State Transfer,
    sort=Representational State Transfer,
    description={\emph{Representational state transfer} (REST) è uno stile architetturale (di architettura software) per i sistemi distribuiti. L'architettura REST si basa su HTTP. Il funzionamento prevede una struttura degli URL ben definita che identifica univocamente una risorsa o un insieme di risorse e l'utilizzo dei metodi HTTP specifici per il recupero di informazioni (GET), per la modifica (POST, PUT, PATCH, DELETE) e per altri scopi (OPTIONS, ecc.)}
}

\newglossaryentry{dbmsg}
{
    name=\glslink{dbms}{DBMS},
    text=Database Management System,
    sort=Database Management System,
    description={In informatica, un \emph{Database Management System} (abbreviato in DBMS o Sistema di gestione di basi di dati) è un sistema software progettato per consentire la creazione, la manipolazione e l'interrogazione efficiente di database.}
}

\newglossaryentry{localstorageg}
{
    name=\glslink{localstorage}{Local Storage},
    text=Local Storage,
    sort=Local Storage,
    description={Il \textit{Local Storage} fornisce accesso all'interfaccia del W3C Web Storage. Quest'ultima fornisce alle applicazioni web metodi e protocolli per l'archiviazione dei dati lato client. L'archiviazione Web supporta l'archiviazione dei dati persistenti, simile ai cookie ma con una capacità notevolmente migliorata}
}

\newglossaryentry{cookieg}
{
    name=\glslink{cookie}{Cookie},
    text=Cookie,
    sort=Cookie,
    description={Gli HTTP cookie sono un tipo particolare di magic cookie (una sorta di gettone identificativo) e vengono utilizzati dalle applicazioni web lato server per archiviare e recuperare informazioni a lungo termine sul lato client}
}

\newglossaryentry{open-sourceg}
{
    name=\glslink{opensource}{open source},
    text=open source,
    sort=open source,
    description={Con open source (in italiano sorgente aperto), in informatica, si indica un tipo di software o il suo modello di sviluppo o distribuzione. Un software open source è reso tale per mezzo di una licenza attraverso cui i detentori dei diritti favoriscono la modifica, lo studio, l'utilizzo e la redistribuzione del codice sorgente}
}

\newglossaryentry{licenza-mit-g}
{
    name=\glslink{licenzamit}{licenza MIT},
    text=licenza MIT,
    sort=licenza MIT,
    description={La Licenza MIT (MIT License in inglese) è una licenza di software libero creata dal Massachusetts Institute of Technology (MIT)}
}

\newglossaryentry{frameworkg}
{
    name=\glslink{framework}{framework},
    text=framework,
    sort=framework,
    description={Un framework, termine della lingua inglese che può essere tradotto come struttura o quadro strutturale, in informatica e specificamente nello sviluppo software, è un'architettura logica di supporto (spesso un'implementazione logica di un particolare design pattern) sulla quale un software può essere progettato e realizzato, spesso facilitandone lo sviluppo da parte del programmatore}
} % database di termini
\makeglossaries


%**************************************************************
% Impostazioni di graphicx
%**************************************************************
\graphicspath{{immagini/}} % cartella dove sono riposte le immagini


%**************************************************************
% Impostazioni di hyperref
%**************************************************************
\hypersetup{
    %hyperfootnotes=false,
    %pdfpagelabels,
    %draft,	% = elimina tutti i link (utile per stampe in bianco e nero)
    colorlinks=true,
    linktocpage=true,
    pdfstartpage=1,
    pdfstartview=,
    % decommenta la riga seguente per avere link in nero (per esempio per la stampa in bianco e nero)
    %colorlinks=false, linktocpage=false, pdfborder={0 0 0}, pdfstartpage=1, pdfstartview=FitV,
    breaklinks=true,
    pdfpagemode=UseNone,
    pageanchor=true,
    pdfpagemode=UseOutlines,
    plainpages=false,
    bookmarksnumbered,
    bookmarksopen=true,
    bookmarksopenlevel=1,
    hypertexnames=true,
    pdfhighlight=/O,
    %nesting=true,
    %frenchlinks,
    urlcolor=webbrown,
    linkcolor=RoyalBlue,
    citecolor=webgreen,
    %pagecolor=RoyalBlue,
    %urlcolor=Black, linkcolor=Black, citecolor=Black, %pagecolor=Black,
    pdftitle={\myTitle},
    pdfauthor={\textcopyright\ \myName, \myUni, \myFaculty},
    pdfsubject={},
    pdfkeywords={},
    pdfcreator={pdfLaTeX},
    pdfproducer={LaTeX}
}

%**************************************************************
% Impostazioni di itemize
%**************************************************************
\renewcommand{\labelitemi}{$\ast$}

%\renewcommand{\labelitemi}{$\bullet$}
%\renewcommand{\labelitemii}{$\cdot$}
%\renewcommand{\labelitemiii}{$\diamond$}
%\renewcommand{\labelitemiv}{$\ast$}


%**************************************************************
% Impostazioni di listings
%**************************************************************
\lstset{
    language=[LaTeX]Tex,%C++,
    keywordstyle=\color{RoyalBlue}, %\bfseries,
    basicstyle=\small\ttfamily,
    %identifierstyle=\color{NavyBlue},
    commentstyle=\color{Green}\ttfamily,
    stringstyle=\rmfamily,
    numbers=none, %left,%
    numberstyle=\scriptsize, %\tiny
    stepnumber=5,
    numbersep=8pt,
    showstringspaces=false,
    breaklines=true,
    frameround=ftff,
    frame=single
} 


%**************************************************************
% Impostazioni di xcolor
%**************************************************************
\definecolor{webgreen}{rgb}{0,.5,0}
\definecolor{webbrown}{rgb}{.6,0,0}


%**************************************************************
% Altro
%**************************************************************

\newcommand{\omissis}{[\dots\negthinspace]} % produce [...]

% eccezioni all'algoritmo di sillabazione
\hyphenation
{
    ma-cro-istru-zio-ne
    gi-ral-din
}

\newcommand{\sectionname}{sezione}
\addto\captionsitalian{\renewcommand{\figurename}{Figura}
                       \renewcommand{\tablename}{Tabella}}

\newcommand{\glsfirstoccur}{\ap{{[g]}}}

\newcommand{\intro}[1]{\emph{\textsf{#1}}}

%**************************************************************
% Environment per ``rischi''
%**************************************************************
\newcounter{riskcounter}                % define a counter
\setcounter{riskcounter}{0}             % set the counter to some initial value

%%%% Parameters
% #1: Title
\newenvironment{risk}[1]{
    \refstepcounter{riskcounter}        % increment counter
    \par \noindent                      % start new paragraph
    \textbf{\arabic{riskcounter}. #1}   % display the title before the 
                                        % content of the environment is displayed 
}{
    \par\medskip
}

\newcommand{\riskname}{Rischio}

\newcommand{\riskdescription}[1]{\textbf{\\Descrizione:} #1.}

\newcommand{\risksolution}[1]{\textbf{\\Soluzione:} #1.}

%**************************************************************
% Environment per ``use case''
%**************************************************************
\newcounter{usecasecounter}             % define a counter
\setcounter{usecasecounter}{0}          % set the counter to some initial value

%%%% Parameters
% #1: ID
% #2: Nome
\newenvironment{usecase}[2]{
    \renewcommand{\theusecasecounter}{\usecasename #1}  % this is where the display of 
                                                        % the counter is overwritten/modified
    \refstepcounter{usecasecounter}             % increment counter
    \vspace{10pt}
    \par \noindent                              % start new paragraph
    {\large \textbf{\usecasename #1: #2}}       % display the title before the 
                                                % content of the environment is displayed 
    \medskip
}{
    \medskip
}

\newcommand{\usecasename}{UC}

\newcommand{\usecaseactors}[1]{\textbf{\\Attori Principali:} #1. \vspace{4pt}}
\newcommand{\usecasepre}[1]{\textbf{\\Precondizioni:} #1. \vspace{4pt}}
\newcommand{\usecasepost}[1]{\textbf{\\Postcondizioni:} #1. \vspace{4pt}}
\newcommand{\usecasedesc}[1]{\textbf{\\Descrizione:} #1. \vspace{4pt}}
\newcommand{\usecasescenario}[1]{\textbf{\\Scenario Principale:} #1. \vspace{4pt}}
\newcommand{\usecasealt}[1]{\textbf{\\Scenario Alternativo:} #1. \vspace{4pt}}

%**************************************************************
% Environment per ``namespace description''
%**************************************************************

\newenvironment{namespacedesc}{
    \vspace{10pt}
    \par \noindent                              % start new paragraph
    \begin{description} 
}{
    \end{description}
    \medskip
}

\newcommand{\classdesc}[2]{\item[\textbf{#1:}] #2}
