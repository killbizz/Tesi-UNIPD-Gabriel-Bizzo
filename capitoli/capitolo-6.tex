% !TEX encoding = UTF-8
% !TEX TS-program = pdflatex
% !TEX root = ../tesi.tex

%**************************************************************
\chapter{Conclusioni}
\label{cap:conclusioni}
%**************************************************************

%**************************************************************
\section{Raggiungimento degli obiettivi}
Gli \hyperref[cap:descrizione-stage]{obiettivi prefissati} all'inizio dello stage risultano tutti soddisfatti. \\
Di seguito viene illustrata la tabella riassuntiva.
\begin{longtable}{| p{.50\textwidth} | p{.50\textwidth} |}
\caption{Tabella di riepilogo dello stato degli obiettivi}
\label{tab:stato-obiettivi}
\hline

\textbf{Obiettivo} & \textbf{Stato}\\

\hline

O01     & Soddisfatto \\

\hline

O02     & Soddisfatto \\

\hline

O03     & Soddisfatto \\

\hline

D01     & Soddisfatto \\

\hline

F01     & Soddisfatto \\

\hline

\end{longtable}

%**************************************************************
\section{Analisi del lavoro svolto}
A causa dell'emergenza sanitaria da Covid-19 lo stage è stato effettuato in gran parte da remoto, utilizzando gli appositi \hyperref[sez:strumenti-organizzativi]{strumenti organizzativi} senza riscontrare particolari problemi. Questa modalità di svolgimento del lavoro, anche se avrebbe potuto creare dei potenziali problemi logistici, mi ha permesso di guadagnare maggiore autonomia nell'apprendimento teorico e nell'organizzazione e svolgimento di attività. \\
Per quanto riguarda l'implementazione della piattaforma gli \hyperref[sez:strumenti-smartworking]{strumenti di sviluppo} sono risultati più che sufficienti per portare a termine il lavoro. In particolare ho avuto la possibilità di approfondire l'IDE Visual Studio Code, il quale offre un'enorme quantità di estensioni utili per uno sviluppo software efficace ed efficiente, soprattutto riguardo all'auto-completamento del codice per Typescript. \\
Infine i prodotti attesi al termine del progetto Voting-Online, ovvero la piattaforma di e-Voting e il documento tecnico contenente il confronto tra Angular e React, sono risultati all'altezza delle aspettative da parte del capo progetto.

%**************************************************************
\section{Valutazione personale}
L'attività di stage mi ha permesso di apprendere nuove tecnologie e di approfondirne altre già conosciute. Sono rimasto molto soddisfatto dalla velocità di apprendimento dei \gls{frameworkg} Angular e Spring, i quali mi hanno permesso di apprezzare particolarmente le conoscenze acquisite in vari corsi di studio durante il periodo universitario. Infatti le conoscenze riguardo la programmazione ad oggetti, derivanti dal linguaggio C++ imparato durante il corso di \textit{Programmazione ad Oggetti} e Java imparato durante il corso di \textit{Altri paradigmi di programmazione}, mi hanno permesso un rapido apprendimento di Spring, mentre le conoscenze derivanti dal corso di \textit{Tecnologie Web} mi hanno permesso una buona realizzazione delle maschere \gls{frontend} utilizzando Angular. Un'ulteriore merito va dato inoltre al corso di \textit{Ingegneria del Software} il quale, grazie al progetto didattico, mi ha dato le competenze necessarie al lavoro collaborativo in un team di sviluppo e alla stesura del documento tecnico. \\ \\
In conclusione l'attività di stage presso l'azienda Sync Lab è risultata molto stimolante e costruttiva, permettendomi una crescita personale importante grazie a questo primo assaggio del mondo del lavoro, che risulta molto diverso rispetto a quello accademico. Questa opportunità mi ha permesso di conoscere meglio le mie capacità nell'ambito lavorativo, permettendomi un miglioramento nella gestione delle tempistiche e nella capacità di lavorare in gruppo.


%Spunti: \\
%\begin{itemize}
%    \item autovalutazione sulla velocità di apprendimento di nuove tecnologie date le mie %conoscenze pregresse costruite all'università (ho avuto la possibilità di apprezzare conoscenze %acquisite vs lacunee in alcuni argomenti)
%    \item interesse negli argomenti e riflessione sul futuro (esperienza mi ha fatto %appassionare allo sviluppo web, continuerò su questa strada, voglio provare a fare full-stack)
%    \item arricchimento teorico, pratico, umano
%    \item migliorata capacità di studio in autonomia e gestione del tempo
%\end{itemize}
