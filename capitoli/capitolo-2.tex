% !TEX encoding = UTF-8
% !TEX TS-program = pdflatex
% !TEX root = ../tesi.tex

%**************************************************************
\chapter{Descrizione dello stage}
\label{cap:descrizione-stage}
%**************************************************************

\intro{In questo capitolo è presente la lista degli obiettivi da raggiungere, la pianificazione delle ore di lavoro da effettuare e l'analisi preventiva dei rischi che potevano venire riscontrati durante lo svolgimento dello stage.}\\

%**************************************************************
\section{Obiettivi}
\subsection*{Notazione}
Si farà riferimento ai requisiti secondo le seguenti notazioni:
\begin{itemize}
	\item \textit{O} per i requisiti obbligatori, vincolanti in quanto obiettivo primario richiesto dal committente;
	\item \textit{D} per i requisiti desiderabili, non vincolanti o strettamente necessari,
		  ma dal riconoscibile valore aggiunto;
	\item \textit{F} per i requisiti facoltativi, rappresentanti valore aggiunto non strettamente 
		  competitivo.
\end{itemize}

Le sigle precedentemente indicate saranno seguite da una coppia sequenziale di numeri, identificativo del requisito.

\subsection*{Obiettivi fissati}
Si prevede lo svolgimento dei seguenti obiettivi:
\begin{itemize}
	\item Obbligatori
	\begin{itemize}
		\obiettiviObbligatori
	\end{itemize}
	
	\item Desiderabili 
	\begin{itemize}
		\obiettiviDesiderabili
	\end{itemize}
	
	\item Facoltativi
	\begin{itemize}
		\obiettiviFacoltativi
	\end{itemize} 
\end{itemize}

%**************************************************************
\section{Pianificazione del lavoro}

\subsection*{Pianificazione settimanale}
\prospettoSettimanale

%**************************************************************
\section{Analisi preventiva dei rischi}

Durante la fase di analisi iniziale sono stati individuati alcuni possibili rischi a cui si potrà andare incontro.
Si è quindi proceduto a elaborare delle possibili soluzioni per far fronte a tali rischi.\\

\begin{risk}{Inesperienza tecnologica}
    \riskdescription{alcune tecnologie da utilizzare risultano totalmente o parzialmente sconosciute}
    \risksolution{studio e approfondimento attraverso tutorial online, corsi forniti dall'azienda e progetti di consolidamento}
    \label{risk:hardware-simulator} 
\end{risk}

\begin{risk}{Monitoraggio del lavoro e delle scadenze}
    \riskdescription{per via dell’emergenza sanitaria dettata dal Covid-19, non sarà sempre possibile confrontarsi con il tutor aziendale e potrebbero emergere dei problemi nell'organizzazione del lavoro per seguire il crono-programma}
    \risksolution{creazione di un foglio condiviso e di un repository per dichiarare le attività svolte giornalmente e per la condivisione del relativo codice prodotto. Gli strumenti utilizzati a tale scopo sono descritti nell'{\hyperref[sez:strumenti-smartworking]{apposita sezione}}}
\end{risk}

\begin{risk}{Impossibilità di recarsi in ufficio}
    \riskdescription{per via dell’emergenza sanitaria dettata dal virus Covid-19, risulterà impossibile lavorare ogni giorno nella sede aziendale}
    \risksolution{utilizzo di diversi strumenti per lo "smart working" e la comunicazione asincrona, elencati nella {\hyperref[sez:strumenti-smartworking]{sezione apposita}}}
    \label{risk:hardware-simulator} 
\end{risk}